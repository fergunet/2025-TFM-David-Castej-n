%%%%%%%%%%%%%%%%%%%%%%%%%%%%%%%%%%%%%%%%%%%%%%%%%%%%%%%%%%%%%%%%%%%%%%%%%%%%%%
% Short Sectioned Assignment LaTeX Template Version 1.0 (5/5/12)
% This template has been downloaded from: http://www.LaTeXTemplates.com
% Original author:  Frits Wenneker (http://www.howtotex.com)
% License: CC BY-NC-SA 3.0 (http://creativecommons.org/licenses/by-nc-sa/3.0/)
%%%%%%%%%%%%%%%%%%%%%%%%%%%%%%%%%%%%%%%%%%%%%%%%%%%%%%%%%%%%%%%%%%%%%%%%%%%%%%


\documentclass[11pt, a4paper]{book}

% PAQUETES ------------------------------------------------------------------------------

\usepackage[T1]{fontenc}
\usepackage[utf8]{inputenc}
% Selecciona el español para palabras introducidas automáticamente, p.ej. "septiembre" en la fecha y especifica que se use la palabra Tabla en vez de Cuadro
\usepackage[spanish,es-tabla,es-nodecimaldot]{babel}
\usepackage{amsmath}
\usepackage{amsfonts}
\usepackage{amssymb}
% Manejo de símbolos de cita múltiples
\usepackage{csquotes}
% Figuras e imágenes
\usepackage{graphics, graphicx, float}
% Permite dar estilo a los subtítulos de las figuras (leyendas)
\usepackage{caption}
% Permite sub-leyendas en sub-figuras (figuras dentro de figuras)
\usepackage{subcaption}
% Permite múltiples filas en tablas
\usepackage{multirow}
% Elimina numeración de paginas en blanco
\usepackage{emptypage}
% Permite continuar con la numeración entre listas distintas
\usepackage{enumitem}
% Introduce la bibliografía en el índice
\usepackage[nottoc]{tocbibind}
% Permite continuar con la numeración entre listas distintas
\usepackage{fourier} % Use the Adobe Utopia font for the document
\usepackage{url} % ,href} %para incluir URLs e hipervínculos dentro del texto (aunque hay que instalar href)
% Para incluir el símbolo del euro
\usepackage[gen]{eurosym} 
\usepackage{cite} %para incluir citas del archivo <nombre>.bib
\usepackage{enumerate}
% Permite añadir comentarios multilinea que no se mostrarán en el PDF final
\usepackage{comment}
\usepackage{tabularx}
\usepackage{booktabs}
\usepackage[table,xcdraw]{xcolor}
% Marca en rojo los elementos pendientes (TO-DOs)
\newcommand{\todo}[1]{\textcolor{orange}{TODO: #1}}
% Permite que el contador de footnotes no se reinicie en cada capítulo
\usepackage{chngcntr}
\counterwithout{footnote}{chapter}

% ESTILOS DE PAGINA ------------------------------------------------------------------
\renewcommand{\familydefault}{\sfdefault}
% Modificar encabezamientos y pie de páginas
\usepackage{fancyhdr}
\pagestyle{fancyplain} % Makes all pages in the document conform to the custom headers and footers
\fancyhead[L]{} % Empty left header
\fancyhead[C]{} % Empty center header
% \fancyhead[R]{Francisco David Castejón Soto} % Nombre del autor en el encabezado derecho
\fancyfoot[L]{} % Empty left footer
\fancyfoot[C]{} % Empty center footer
\fancyfoot[R]{\thepage} % Page numbering for right footer
\renewcommand{\headrulewidth}{0pt} % Remove header underlines
\renewcommand{\footrulewidth}{0pt} % Remove footer underlines
\setlength{\headheight}{13.6pt} % Customize the height of the header

\usepackage{titlesec, blindtext, color}
\definecolor{gray75}{gray}{0.75}
\newcommand{\hsp}{\hspace{20pt}}
\titleformat{\chapter}[hang]{\Huge\bfseries}{\thechapter\hsp\textcolor{gray75}{|}\hsp}{0pt}{\Huge\bfseries}
\setcounter{secnumdepth}{4}
\usepackage[Lenny]{fncychap}

% For code snippets
\usepackage{minted}
\usemintedstyle{colorful}
% Control de enlaces hipertexto en el documento. Este paquete debe cargarse el último.
\usepackage{hyperref}
\hypersetup{
	colorlinks=true,	% false: boxed links; true: colored links
	linkcolor=black,	% color of internal links
	urlcolor=cyan		% color of external links
}

% Información del documento ---------------------------------------------------------
\author{Francisco David Castejón Soto}
\title{Entrenamiento de una IA mediante aprendizaje por refuerzo para un juego hecho en Unreal Engine}
\date{\today}

% DOCUMENTO ----------------------------------------------------------------------------
\begin{document}

% Páginas preliminares sin numeración
\frontmatter

\begin{titlepage}
	\thispagestyle{empty}

	\centering
	\includegraphics[width=0.9\textwidth]{logos/logo_ugr.jpg}
	\vspace{1.0cm}

	{\rmfamily\textsc{\Large MÁSTER UNIVERSITARIO}}\\[0.2cm]
	{\rmfamily\textsc{\large en Ciencia de Datos e Ingeniería de Computadores}}
	\vspace{1.5cm}

	\vfill

	{\Huge\bfseries Trabajo de Fin de Máster}\\[0.5cm]
	\rule{\textwidth}{2pt}\\[0.5cm]
	{\Large\bfseries Desarrollo y Optimización de un Agente Inteligente \\
	para Scripts of Tribute mediante Algoritmos Evolutivos}
	\vspace{1.5cm}

	\vfill

	\begin{tabular}{@{}c@{}}
		\textbf{\large Autor}         \\[0.3cm]
		Francisco David Castejón Soto \\[1cm]
		\textbf{\large Director}      \\[0.3cm]
		Dr. Pablo García Sánchez
	\end{tabular}

	\vfill

	\includegraphics[width=0.4\textwidth]{logos/etsiit_logo.png}\\[0.3cm]
	{\rmfamily\textsc{\footnotesize Escuela Técnica Superior de Ingenierías Informática y de Telecomunicación}}\\
	\rule{0.1\textwidth}{0.5pt}\\[0.3cm]
	{\large Granada, 1 Julio de 2025}

\end{titlepage}
\chapter*{Prefacio}

Para la realización de este documento, se ha utilizado la plantilla \LaTeX \cite{guervos_jjplantilla-tfg-etsiit_2024} específica para la ETSIIT.
\chapter*{Agradecimientos} \label{ch:agradecimientos}

A mi familia por el apoyo que me han brindado incondicionalmente durante estos años fuera de casa. Y a mi pareja, por mostrarme más caminos de los que nunca hubiera pensado que existieran para mí.

% Índices
\cleardoublepage
\tableofcontents

\cleardoublepage
\listoffigures

\cleardoublepage
\listoftables

% Contenido principal
\mainmatter

\part{Cartas y Bots} \label{part:idea}
\chapter{Introducción} \label{chap:introduccion}

% Hablar la importancia de la IA en los videojuegos, especialmente en los de estrategia
% Introducir brevemente el concepto de algoritmos evolutivos como técnica para entrenar agentes
% Concluir presentando el problema que se aborda: entrenar un bot para el juego "Tales of Tribute"

La inteligencia artificial (IA) en videojuegos existe en una intersección entre la ciencia computacional y el arte del entretenimiento, dando vida a mundos virtuales y creando oponentes dignos de nuestras mejores estrategias. En este contexto, la IA se refiere únicamente a un conjunto de algoritmos y técnicas diseñadas para cumplir una función muy específica dentro del videojuego. Ejemplos de esto son la IA que ``conduce'' los coches en un juego de carreras, los Pokémon contra los que el jugador se enfrenta en una batalla o el simple movimiento de una línea de píxeles (a modo de pala contrincante) en el videojuego Pong. Esta acepción específica contrasta con la idea más extendida de IA, que posee un conjunto de definiciones más amplias y generalistas como ``la máquina que aprende'' o ``el estudio y construcción de sistemas que hacen <<lo correcto>>, en función de su objetivo'' \cite{russell_artificial_2021}.

Independientemente de la definición que se utilice, lo cierto es que un factor clave en el desarrollo de un videojuego es la creación de una IA que resulte entretenida para el jugador. De la misma forma que en un libro o una película es necesaria una trama que plantee un desafío, como la lucha contra el Imperio Galáctico en Star Wars, o un compañero de aventuras que ayude al protagonista, como Sam en El Señor de los Anillos, en un videojuego es necesaria una IA que controle esos elementos que interactúan con el jugador, ya sean enemigos, aliados o incluso el propio entorno. Desde los propios inicios de la industria del videojuego, donde se construían máquinas específicas para ejecutar un juego concreto, como es el caso de Nim en 1948 \cite{redheffer_machine_1948}, hasta los videojuegos modernos, donde se utilizan técnicas de IA más complejas, como el aprendizaje por refuerzo (más en la sección \ref{sec:estado_arte}), la IA ha sido un componente esencial para crear experiencias de juego atractivas y desafiantes.

La intención de este trabajo es la de crear una IA contra la que jugar en un videojuego de estrategia. En concreto, la IA que controla a un agente autónomo (o simplemente ``bot'') en el videojuego de cartas \textit{Tales of Tribute}. Es importante la distinción entre ``juego'' y ``videojuego'' en este caso, ya que las cartas no son físicas, sino que únicamente existen en el universo digital del videojuego. Para guiar las decisiones del bot, un algoritmo evolutivo se encarga de ajustar los pesos que se utilizan para calcular la puntuación de cada jugada posible, eligiendo vorazmente la jugada que maximiza dicha puntuación. Así se ha conseguido crear un contrincante que juega de forma competitiva tanto contra el jugador humano como contra otros bots manejados por IAs totalmente diferentes.
\chapter{Objetivos}


\section{Objetivo principal}


\section{Objetivos generales}


\section{Objetivos específicos}



\chapter{Antecedentes}

\section{Inteligencia artificial en videojuegos}
%% Hablar sobre las técnicas comunes de IA en juegos (máquinas de estados, árboles de comportamiento, MCTS)


\section{Estado del arte}
% Qué técnicas se están usando actualmente

\section{Trabajos relacionados}
% Paper de Pablo y TFG de los ganadores de la competición de IA de Tales of Tribute.
% Comparar sus enfoques al mio
\chapter{Recursos}


\section{Recursos Software}


\section{Recursos Hardware}


\section{Recursos Humanos}



\chapter{Distribución temporal}


\section{Cronograma}


\part{Sofware y preparación de los experimentos} \label{part:preparacion}
\chapter{Algoritmos}


\section{Toma de decisiones del bot}


\section{Ajuste de los pesos del bot}


\subsection{Los algoritmos evolutivos}


\subsection{La evolución canónica}


\subsection{Los pesos del bot}
\chapter{Sofware desarrollado}


\section{Bot en C\# para Scripts of Tribute}

\subsection{El cálculo de la mejor acción}

\subsection{Más orientación para los casos límite}

\section{Algoritmos evolutivos en Python con Inspyred}

\subsection{Parámetros del algoritmo}

\subsection{Conexión entre el entrenador y el bot}

\subsection{Paralelización de procesos en Python}

\subsection{El salón de la fama}

\part{Experimentos y resultados} \label{part:resultados}
\chapter{Diseño experimental} \label{chap:experimentacion}
% Como se prueban los bots

\todo{Probar diferentes combinaciones en el modo hibrido.}
\todo{Hablar sobre los individuos generalistas (todos los pesos parecidos) y especialistas (0-1 en los pesos) en los mejores pesos.}
\todo{Ver la diferencia entre usar el hall of fame y no usarlo ya que no debería añadir info de una train run a otra. Mirar en la literatura que es exactamente hall of fame.}
\todo{Para comparar los modos de entrenamiento, intentar que todos los algoritmos se ejecuten durante la misma cantidad de tiempo (aunque tengan tamaños de poblaciones diferentes).}
\todo{Lanzar sin el hall of fame para evitar las diferencias entre runs.}

\section{Configuración de los experimentos} \label{sec:configuracion_experimentos}
% Entrenamiento en modo Coevolución
% Entrenamiento en modo Híbrido con varias organizaciones
% Parámetros utilizados de configuración


\section{Métricas de evaluación} \label{sec:metricas_evaluacion}
% Evolución del fitness (máximo, medio, mínimo) a lo largo de las generaciones
% Convergencia y distribución de los pesos de la población
% Rendimiento del "campeón" de cada `train_run` en el benchmark final y contra otros bots fuera del entrenador
\chapter{Resultados y discusión} \label{chap:resultados}
% Interpretar los resultados de cómo y por qué
% - ¿Por qué el modo híbrido funciona mejor que los otros?
% - La tendencia de los pesos a irse a 0 o 1
% - Limitaciones del enfoque: bot sin memoria ni planificación, etc

\section{A nivel de población} \label{sec:a_nivel_de_poblacion}


\subsection{Evolución del fitness} \label{sec:evolucion_fitness_poblacion}


\subsection{Pesos medios} \label{sec:pesos_medios_poblacion}


\subsection{Pesos a lo largo de las generaciones} \label{sec:pesos_a_lo_largo_generaciones_poblacion}


\subsection{Variación de los pesos finales} \label{sec:variacion_pesos_finales_poblacion}


\section{A nivel de líderes} \label{sec:a_nivel_de_lideres}


\subsection{Evolución del fitness} \label{sec:evolucion_fitness_lideres}


\subsection{Pesos medios} \label{sec:pesos_medios_lideres}


\subsection{Pesos a lo largo de las generaciones} \label{sec:pesos_a_lo_largo_generaciones_lideres}


\subsection{Variación de los pesos finales} \label{sec:variacion_pesos_finales_lideres}





% https://en.wikipedia.org/wiki/AlphaStar_(software)
% https://www.bbc.com/news/technology-50212841
% Unlike AlphaZero, AlphaStar initially learns to imitate the moves of the best players in its database of human vs. human games; this step is necessary to solve what DeepMind's Dave Silver calls "the exploration problem": discovering new strategies would otherwise be like finding a "needle in a haystack". Agents then play each other and deploy deep reinforcement learning. These main agents also learn by playing against suboptimal "exploiter agents" whose purpose is to expose weaknesses in the main agents.
\chapter{Discusión de los resultados}

\chapter{Conclusiones} \label{chap:conclusiones}


\section{Objetivos alcanzados} \label{sec:objetivos_alcanzados}
% Resumen los hallazgos
% Explicar punto por punto qué objetivos se han cumplido


\section{Líneas de trabajo futuro} \label{sec:trabajo_futuro}
% Probar operadores de mutación y cruce más sofisticados
% Expandir el conjunto de oponentes fijos o dinámicos
% Implementar estrategias de coevolución más complejas

% Material de referencia
\backmatter

% % Bibliografía
\cleardoublepage
\bibliographystyle{plainurl}
\bibliography{bibliografia}

% \appendix
% \input{apendices/codigo}

\end{document}