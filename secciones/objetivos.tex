\chapter{Objetivos} \label{chap:objetivos}
% Definir lo que se busca conseguir con el TFM


\section{Objetivo principal} \label{sec:objetivo_principal}
% Desarrollar un bot
% Portfolio para la búsqueda de empleo

% https://ieeexplore.ieee.org/document/585893
% https://link.springer.com/article/10.1007/s13748-014-0051-8
% https://es.wikipedia.org/wiki/Dise%C3%B1o_de_juegos

El objetivo principal de este Trabajo de Fin de Máster se puede dividir en dos partes. La primera es ahondar en la investigación y entendimiento de un area tan compleja como las metaheurísticas, y en concreto los algoritmos evolutivos. La optimización de soluciones a problemas complejos mediante técnicas que tratan incluir conocimiento específico del dominio conjuntamente con la generación iterativa de otras soluciones parciales es un campo de estudio realmente útil. Aunque el teorema de ``No Free Lunch'' nos advierte de que no existe una única técnica que sea la mejor para todos los problemas (\todo{cita nfl}) (y es cierto que este tipo de técnicas tienen sus), la experiencia ha demostrado que los algoritmos evolutivos son una herramienta que se puede aplicar a prácticamente cualquier tipo de problema de optimización (\todo{cita evolutivos}).

Es precisamente esa versatilidad la que ha propiciado su uso durante el desarrollo del segundo objetivo principal de este proyecto: la creación de un bot para un videojuegos de cartas. Este tipo de videojuegos cuentan con un inmenso número de variables, pues no solo se deben tener en cuenta las cartas existentes en las manos de cada jugador y en la pila de cartas, sino también las mecánicas\footnote{Gran parte de lo que define un juego es sus mecánicas, sus reglas, lo que los jugadores deben llevar a cabo para ganar. La labor del creador de videojuegos es conseguir un conjunto de reglas que estén equilibradas y permitan disfrutar a los jugadores (\todo{cita mecanicas}).} intrínsecas del juego, como la vida, los recursos de compra, la posibilidad de recuperar cartas usadas, o el uso de otro tipo de habilidades especiales. Todo esto hace que la creación de un bot que juegue de forma competitiva sea un reto interesante, y que el uso de técnicas de optimización evolutiva sea una herramienta adecuada para conseguirlo.

\section{Objetivos específicos} \label{sec:objetivos_especificos}

% Objetivos del formulario de petición de tema:
\begin{itemize}
	\item OG1: Analizar en profundidad las mecánicas del juego ``Tales of Tribute'', el entorno de desarrollo ``Scripts of Tribute'' y adquirir las competencias necesarias en el lenguaje de programación C\#.
	\item OG2: Diseñar e implementar un agente inteligente en C\# para ``Scripts of Tribute'', cuya toma de decisiones se base en una evaluación heurística del estado del juego mediante una función de fitness ponderada, incorporando conocimiento experto del dominio.
	\item OG3: Desarrollar un marco de optimización en Python para ajustar los pesos de la función de fitness del agente, implementando y comparando dos estrategias principales: algoritmos coevolutivos y entrenamiento supervisado contra agentes de referencia de ``Scripts of Tribute''.
	\item Desarrollar un conjunto de herramientas para la visualización y análisis de los datos generados durante el entrenamiento.
	\item OG5: Evaluar cuantitativamente el rendimiento del agente entrenado mediante las diferentes estrategias, utilizando métricas relevantes como la tasa de victorias contra distintos oponentes.
	\item OG6: Analizar comparativamente la efectividad y eficiencia de las estrategias de optimización implementadas (coevolución vs. entrenamiento contra referentes), discutiendo sus ventajas y desventajas en el contexto específico de ``Scripts of Tribute''.
	\item OG7: Investigar y contextualizar el enfoque desarrollado frente a otras técnicas predominantes en competiciones similares de IA en juegos, como Monte Carlo Tree Search (MCTS), analizando las razones de su éxito en ediciones anteriores.
\end{itemize}