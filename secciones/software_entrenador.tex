\chapter{Sistema de entrenamiento evolutivo}
% La parte en Python

\section{Arquitectura general del entrenador}
% Visión general del sistema. Script en Python que usa Inspyred para orquestar un algoritmo evolutivo
% Encontrar el conjunto de pesos óptimo para el bot
% Diagrama de flujo

\section{Comunicación entrenador-imulador}
% Los subprocesos y las variables de entorno
% os.environ para pasar los pesos de un individuo al bot en C#

\section{El algoritmo evolutivo con Inspyred}
% Justificar la elección de Inspyred
% Describir los componentes de un EA: individuos, población, generación inicial, operadores de variación, evaluación del fitness, selección, etc
% Explicar los valores por defecto de Inspyred

\section{Paralelización del algoritmo}
% Uso de multiprocessing para el diccionario y concurrent.futures para el evaluador

\section{Mecanismos de evaluación del fitness}

\subsection{Modo fijo: evaluación contra oponentes estáticos}
% El individuo juega contra un conjunto predefinido de bots. El fitness es el número de victorias
% inspyred.ec.evaluators.parallel_evaluation_mp
\subsection{Modo coevolución: competición interna}
% Los individuos (padres e hijos) compiten entre sí en un torneo round robin. El fitness es el número de victorias
% evaluate_coevolution_orchestrator
\subsection{Modo híbrido: combinando estrategias}
% Sistema de entrenamiento por segmentos hybrid_schedule_str
% Gestión de cambios de modo
\section{El salón de la fama}
% Mantener un diccionario de oponentes de élite (catastrophic forgetting)