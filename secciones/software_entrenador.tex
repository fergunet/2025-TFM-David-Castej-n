\chapter{Sistema de entrenamiento evolutivo} \label{chap:entrenador}
% La parte en Python


\section{Arquitectura general del entrenador} \label{sec:arquitectura_entrenador}
% Visión general del sistema. Script en Python que usa Inspyred para orquestar un algoritmo evolutivo
% Encontrar el conjunto de pesos óptimo para el bot
% Diagrama de flujo


\section{Comunicación entrenador-imulador} \label{sec:comunicacion_entrenador_simulador}
% Los subprocesos y las variables de entorno
% os.environ para pasar los pesos de un individuo al bot en C#


\section{El algoritmo evolutivo con Inspyred} \label{sec:algoritmo_evolutivo_inspyred}
% Justificar la elección de Inspyred
% Describir los componentes de un EA: individuos, población, generación inicial, operadores de variación, evaluación del fitness, selección, etc
% Explicar los valores por defecto de Inspyred


\section{Paralelización del algoritmo} \label{sec:paralelizacion_algoritmo}
% Uso de multiprocessing para el diccionario y concurrent.futures para el evaluador


\section{Mecanismo de evaluación del fitness} \label{sec:mecanismos_evaluacion_fitness}
% El conteo de victorias como métrica de fitness

\subsection{Modo fijo: evaluación contra oponentes estáticos} \label{sec:modo_fijo_evaluacion}
% El individuo juega contra un conjunto predefinido de bots. El fitness es el número de victorias
% inspyred.ec.evaluators.parallel_evaluation_mp


\subsection{Modo coevolución: competición interna} \label{sec:modo_coevolucion_competicion}
% Los individuos (padres e hijos) compiten entre sí en un torneo round robin. El fitness es el número de victorias
% evaluate_coevolution_orchestrator


\subsection{Modo híbrido: combinando estrategias} \label{sec:modo_hibrido_combinando}
% Sistema de entrenamiento por segmentos hybrid_schedule_str
% Gestión de cambios de modo


\section{El salón de la fama} \label{sec:salon_fama}
% Mantener un diccionario de oponentes de élite (catastrophic forgetting)