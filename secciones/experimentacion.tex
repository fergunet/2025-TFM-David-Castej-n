\chapter{Diseño experimental} \label{chap:experimentacion}
% Como se prueban los bots

\todo{Probar diferentes combinaciones en el modo hibrido.}
\todo{Hablar sobre los individuos generalistas (todos los pesos parecidos) y especialistas (0-1 en los pesos) en los mejores pesos.}
\todo{Ver la diferencia entre usar el hall of fame y no usarlo ya que no debería añadir info de una train run a otra. Mirar en la literatura que es exactamente hall of fame.}
\todo{Para comparar los modos de entrenamiento, intentar que todos los algoritmos se ejecuten durante la misma cantidad de tiempo (aunque tengan tamaños de poblaciones diferentes).}
\todo{Lanzar sin el hall of fame para evitar las diferencias entre runs.}

\section{Configuración de los experimentos} \label{sec:configuracion_experimentos}
% Entrenamiento en modo Coevolución
% Entrenamiento en modo Híbrido con varias organizaciones
% Parámetros utilizados de configuración


\section{Métricas de evaluación} \label{sec:metricas_evaluacion}
% Evolución del fitness (máximo, medio, mínimo) a lo largo de las generaciones
% Convergencia y distribución de los pesos de la población
% Rendimiento del "campeón" de cada `train_run` en el benchmark final y contra otros bots fuera del entrenador