\chapter{Introducción} \label{chap:introduccion}

La inteligencia artificial (IA) en videojuegos existe en una intersección entre la ciencia computacional y el arte del entretenimiento, dando vida a mundos virtuales y creando oponentes dignos de nuestras mejores estrategias. En este contexto, la IA se refiere únicamente a un conjunto de algoritmos y técnicas diseñadas para cumplir una función muy específica dentro del videojuego. Algunos ejemplos serían la IA que ``conduce'' los coches en un videojuego de carreras, los \textit{Pokémon} contra los que el jugador se enfrenta en una batalla o el simple movimiento de una línea de píxeles (la paleta del oponente controlado por la máquina) en el videojuego \textit{Pong}. Esta acepción específica contrasta con la idea más extendida de IA, la cual posee un conjunto de connotaciones más amplias y generalistas como ``la máquina que aprende'' o ``el estudio y construcción de sistemas que hacen <<lo correcto>>, en función de su objetivo'' \cite{russell_artificial_2020}. Independientemente de la definición que se utilice, un factor clave en el desarrollo de un videojuego es la creación de una IA que resulte entretenida para el jugador. De la misma forma que en un libro o una película, es necesaria una trama que plantee un desafío, por ejemplo la lucha contra el Imperio Galáctico en \textit{Star Wars}; o un compañero de aventuras que ayude al protagonista, como Sam en \textit{El Señor de los Anillos}. En un videojuego es necesaria una IA que controle esos elementos que interactúan con el jugador, ya sean enemigos, aliados o incluso el propio entorno. Desde los inicios de la industria, cuando se diseñaban máquinas específicas para un solo juego como \textit{Nim} en 1948 \cite{redheffer_machine_1948}, hasta los videojuegos modernos que utilizan complejas técnicas de aprendizaje automático como el aprendizaje por refuerzo \cite{gaudreau_game_2025}, la IA ha demostrado ser un componente clave para crear experiencias de juego divertidas y desafiantes.

La intención de este trabajo es la de crear una IA contra la que jugar en un videojuego de estrategia. En concreto, la que controla a un agente autónomo (o simplemente ``bot'') en un simulador del videojuego de cartas \textit{Tales of Tribute}\footnote{Es importante la distinción entre ``juego'' y ``videojuego'' en este caso, ya que las cartas no son físicas y tangibles, sino que únicamente existen en el universo digital del videojuego.}. Para guiar las decisiones del bot, se ha empleado un algoritmo evolutivo que se encarga de ajustar los pesos que se utilizan para calcular la puntuación de cada posible jugada, eligiendo vorazmente la jugada que maximiza dicha puntuación. De esta forma se ha conseguido crear un contrincante que juega de forma competitiva tanto contra el jugador humano como contra otros bots manejados por algoritmos de inteligencia artificial totalmente diferentes.