\chapter{Antecedentes y estado del arte} \label{chap:antecedentes}

\section{Inteligencia artificial en videojuegos comerciales} \label{sec:ia_videojuegos}
% Hablar sobre las técnicas comunes de IA en juegos (máquinas de estados, árboles de comportamiento, MCTS)
% Qué técnicas se están usando actualmente
% Metaheurísticas y algoritmos evolutivos en videojuegos

En esta sección se realiza un breve repaso a las técnicas de inteligencia artificial más utilizadas a lo largo de la historia de los videojuegos, así como algunos ejemplos del estado del arte en este ámbito. Los usos de estas ténicas no solo abarcan su función más habitual de antagonista o compañero, sino que también abarca otros aspectos como la generación de contenido procedural o la deformación realista de las mallas \footnote{Un ``mesh'', o malla en español, es una estructura tridimensional formada por una colección de vértices, aristas y caras que definen la forma de un objeto 3D dentro de un videojuego \cite{universidad_europea_que_2025}.} de los personajes animados.

\subsection{Funciones hash}

Una de las técnicas más sencillas y antiguas para controlar partes específicas de un videojuego son las funciones hash. Este tipo de funciones simplemente reciben un conjunto de parámetros como entrada y devuelven un valor o acción a realizar. Dada su simplicidad, generan comportamientos predecibles, pero requieren de muy pocos recursos para su procesamiento y son fáciles de implementar. Un ejemplo clásico de IA basada en funciones hash es el en juego \textit{Space Invaders} \cite{wikipedia_artificial_2025} donde los invasores estaban controlados con funciones hash. 

\subsection{Máquinas de estados finitos}

Las máquinas de estados finitos son un modelo de computación conceptual que describe un sistema que solo puede encontrarse en un estado a la vez, y que puede cambiar a uno de sus otros estados como respuesta a ciertos eventos. Este mecanismo se ha utilizado y sigue utilizándose para un gran número de aplicaciones dentro de los videojuegos. Por ejemplo, en el juego \textit{Pac\-man}, la IA de los fantasmas se rije por el estado en el que se encuentran, como "cazando" o "siendo cazados" \cite{mike_game_2016}. Otro uso está en la gestión de las animaciones de los personajes, donde cada estado corresponde a una posición o movimiento específico de la malla, que al mezclarse mediante interpolaciones generan animaciones. Al igual que las funciones hash, requieren de pocos recursos para su procesamiento, pero su naturaleza determinista y la posibilidad de entrar en bucles sin salida si no están bien diseñadas, limitan su uso en los escenarios más exigentes.

\subsection{Árboles de comportamiento}

Como evolución a las máquinas de estados finitos, surgieron los árboles de comportamiento, permitiendo un mayor grado de complejidad y flexibilidad en la toma de decisiones de los personajes. Los árboles de comportamiento son grafos dirijidos acíclicos, con nodos hoja que representan acciones y varios tipos de control de flujo que determinan el orden de ejecución de las hojas \cite{epic_games_behavior_2025}. Una de sus características más destacadas es su modularidad, lo que permite reutilizar y combinar comportamientos fácilmente. Un fantastico ejemplo de su correcto uso se expuso en la charla de la ``Game Developers Conference'' (GDC) de 2005, sobre la IA de \textit{Halo 2}, donde se explica que su capacidad de reutilización de árboles y el ser más sencillos de entender para los diseñadores del juego, hizo que los programadores optaran por su implementación en lugar de las máquinas de estados finitos \cite{isla_managing_2005}.


\subsection{Planificación de acciones orientada a objetivos}

En contraste con los árboles de comportamiento, que generan una jerarquía de caminos, la planificación de acciones orientada a objetivos (GOAP, por sus siglas en inglés) es más parecida a la planificación de rutas, donde se busca el camino más óptimo para alcanzar un objetivo en función del estado del mundo. Cada uno de los pasos que llevan al objetivo tiene una serie de precondiciones que deben cumplirse, a los cuales se les puede asignar un coste. Es tal su similitud con la planificación de rutas, que este tipo de IA suelen utilizar internamente algoritmos de búsqueda como A*\footnote{A* también se utiliza en la búsqueda de caminos tradicional dentro de los videojuegos, es decir, para saber la ruta que debe tomar un objeto por el mapa para llegar a un destino, y no solo en GOAP.} para encontrar el camino más óptimo. Una de las mejores implementaciones de este tipo de IA se encuentra en el título de 2005 \textit{F.E.A.R.}, donde los enemigos y aliados utilizan GOAP para ``planificar'' su comportamiento una manera mucho más proactiva que sus predecesores \cite{jeff_gdc_2006}.


\subsection{Aprendizaje por refuerzo y redes neuronales}

El aprendizaje por refuerzo (RL, por sus siglas en inglés) es un subcampo del aprendizaje automático que trata de mejorar el comportamiento de un agente a través del feedback que recibe de su entorno. Utilizando un sistema de recompensas y penalizaciones, el agente aprende a tomar decisiones que maximicen su recompensa total a lo largo del tiempo. Aunque el RL empezó con técnicas tabulares, los algoritmos recientes incorporan redes neuronales en diferentes partes del proceso \cite{ghasemi_comprehensive_2025}. Quizás el ejemplo que más se suele ver en la literatura de videojuegos es el uso de RL para controlar vehículos autónomos. Una implementación de esta tecnología en un videojuego comercial está en el título \textit{Star Wars: Outlaws}, donde las motos controladas por IA utilizan un sistema de RL para moverse por el mapa o perseguir al jugador \cite{gaudreau_game_2025}. Sus desarrolladores argumentan que les permitió tener un sistema de control que se adapta a las diferentes rutas del juego sin importar los cambios en el mapa que haya entre versiones.

Pero el uso de redes neuronales no se limita a la toma de decisiones. Su capacidad para realizar inferencias rápidas a partir de datos complejos las hace ideales para otras tareas, como la deformación de mallas. El objetivo en este caso es evitar el fenómeno conocido como el ``valle inquietante'' (\textit{Uncanny Valley}), que se produce cuando una réplica de un ser humano es muy realista, pero no perfecta, generando una sensación de extrañeza en el espectador. Un personaje que sonríe, por ejemplo, pero cuyos músculos faciales alrededor de los ojos permanecen inmóviles, es un caso clásico de este efecto. Para intentar superar este problema, Epic Games optó por crear un sistema para Unreal Engine que utiliza modelos de redes neuronales que simulan el comportamiento de la musculatura y los tejidos blandos bajo la piel al extenderse y contraerse \cite{epic_games_ml_2025}. Un ejemplo de uso de esta reciente tecnología se puede encontrar en la demo técnica que CD Projekt Red mostró en el Unreal Fest de 2025 \cite{cd_projekt_red_witcher_2025}. En ella, se mostraba como se modelaba internamente la musculatura de un caballo para que sus animaciones de movimiento fueran mucho más realistas.

% Me parece que ya es suficiente con lo que hay. Quito esta sección.
% \subsection{Unión de varias técnicas}
% The Last of Us: Parte II --> muchos sistemas entrelazados
% Alien: Isolation --> Árbol de comportamiento + Director de IA

% También quito esta sección, no es una tecnología lo suficientemente madura como para uso actual en videojuegos comerciales.
% \subsection{Modelos de lenguaje grande}

\section{Metaheurísticas en videojuegos} \label{sec:estado_arte}
\subsection{Generación de contenido procedural}


\subsection{El desafío de las metaheurísticas en videojuegos comerciales}


\section{Algoritmos evolutivos en juegos de cartas} \label{sec:trabajos_relacionados}
% Paper de Pablo y TFG de los ganadores de la competición de IA de Tales of Tribute. Comparar sus enfoques al mio


\subsection{Optimización de agentes de Hearthstone mediante un algoritmo evolutivo}


\subsection{Desarrollo de un agente para el concurso de IA Tales of Tribute}
% Comprobar si los del MCTS usan algoritmos evolutivos
\todo{Analizar la estrategia de MCTS de los ganadores de la competición de IA de Tales of Tribute.}