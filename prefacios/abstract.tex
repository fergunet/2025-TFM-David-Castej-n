\chapter*{Resumen}

{\large\textbf{Resumen}}

En el ámbito de la inteligencia artificial para videojuegos, la creación de agentes autónomos que actúen como jugadores virtuales sigue siendo un desafío considerable. Este trabajo aborda dicho desafío en el contexto del motor de simulación \textit{Scripts of Tribute} para el juego de cartas \textit{Tales of Tribute}, presentando el diseño y la implementación de un sistema completo para el entrenamiento de un bot competitivo mediante técnicas evolutivas.

La metodología se articula en dos componentes principales. Primero, se desarrolló un agente cuya toma de decisiones se basa en una función de evaluación heurística, parametrizada por un vector de pesos que pondera distintas facetas del estado del juego. Segundo, se construyó un sistema de entrenamiento que utiliza un algoritmo de estrategia evolutiva, para optimizar dicho vector de pesos. Para ello, se implementaron y compararon diversas arquitecturas de evaluación, incluyendo un modo fijo contra oponentes estáticos, un modo de coevolución para fomentar la competición interna, y un modo híbrido que combina ambas estrategias. Adicionalmente, se integró un mecanismo de ``salón de la fama'' para preservar a los individuos de élite y combatir el olvido catastrófico.

Los resultados experimentales demuestran una clara especialización de los agentes en función del modo de entrenamiento. Además, el salón de la fama implementado es beneficioso en función de la estabilidad del objetivo de la población. En última instancia, se concluye que el sistema propuesto es capaz de generar agentes competitivos, validando la eficacia del enfoque evolutivo para la optimización de la toma de decisiones en juegos de estrategia.

\vspace{\baselineskip}
\noindent\textbf{Palabras clave:} Inteligencia Artificial para Videojuegos, Algoritmos Evolutivos, Estrategias Evolutivas, Coevolución, Juegos de Cartas.

\newpage
{\large\textbf{Abstract}}

\vspace{0.5\baselineskip}

\textit{In the field of artificial intelligence for video games, the creation of autonomous agents that act as virtual players remains a considerable challenge. This report addresses that challenge in the context of the simulation engine \textit{Scripts of Tribute} for the card game \textit{Tales of Tribute}, presenting the design and implementation of a complete system for training a competitive bot using evolutionary techniques.}

\textit{The methodology is articulated in two main components. First, an agent was developed whose decision making is based on a heuristic evaluation function, parameterized by a vector of weights that weights different facets of the game state. Second, a training system using an evolutionary strategy algorithm was built to optimize this vector of weights. For this purpose, several evaluation architectures were implemented and compared, including a fixed mode against static opponents, a co-evolution mode to encourage internal competition, and a hybrid mode combining both strategies. Additionally, a ``hall of fame'' mechanism was integrated to preserve elite individuals and combat catastrophic forgetting.}

\textit{The experimental results demonstrate a clear specialization of the agents as a function of the training mode. Moreover, the implemented hall of fame is beneficial as a function of the stability of the population objective. Ultimately, it is concluded that the proposed system is able to generate competitive agents, validating the effectiveness of the evolutionary approach for the optimization of decision making in strategy games.}

\vspace{\baselineskip}
\noindent\textbf{Keywords:} Artificial Intelligence for Video Games, Evolutionary Algorithms, Evolution Strategies, Coevolution, Card Games.